%\begin{center}
%\large \bf \runtitle
%\end{center}
%\vspace{1cm}
\chapter*{\runtitulo}
\noindent With the growing interest in Science and Technology in recent years, teachers face the challenge of correcting large volumes of code while keeping track of the knowledge level achieved by different groups of students. In this context, having tools that allow early prediction of dropout risk in introductory programming courses becomes crucial to focus attention on those students who need it most.

For this purpose, a dataset corresponding to an Introduction to Python Programming course was analyzed. Only their code productions were studied, transformed into embeddings with the help of the \emph{CodeBERT} model. Using dimensionality reduction techniques and, subsequently, clustering, it was possible to characterize some of the students who would drop out of the course early. In particular, the group of students who would abandon the course upon reaching the first evaluation instance. For the proposed exercises, it was not possible to differentiate between those students who would complete the course and those who continued with submissions beyond the first exam.
\bigskip

\noindent\textbf{Keywords:} Clustering, CodeBERT, Embeddings, Teaching, Python, Programming.