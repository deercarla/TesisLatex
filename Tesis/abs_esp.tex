%\begin{center}
%\large \bf \runtitulo
%\end{center}
%\vspace{1cm}
\chapter*{\runtitulo}

\noindent Con el creciente interés en la Ciencia y la Tecnología en los últimos años, los docentes enfrentan el desafío de corregir grandes volúmenes de código sin perder de vista el nivel de conocimiento alcanzado por los diferentes grupos de estudiantes. En este contexto, contar con herramientas que permitan predecir de manera temprana el riesgo de abandono en cursos introductorios de programación se vuelve crucial para enfocar la atención en aquellos alumnos que más lo necesitan.

Para ello se analizó  un conjunto de datos correspondiente a un curso de Introducción a la Programación en Python. Se estudiaron exclusivamente sus producciones de código, transformadas a \emph{embeddings} con ayuda del modelo \emph{CodeBERT}. Utilizando técnicas de reducción de la dimensionalidad y, posteriormente, \emph{clustering} se logró caracterizar a parte de los alumnos que abandonarían el curso de manera temprana. En particular, al grupo de estudiantes que abandonarían el curso una vez llegada la primera instancia de evaluación. No se logró diferenciar, para los ejercicios propuestos, entre aquellos alumnos que finalizarían el curso y aquellos que prosiguieron con las entregas más allá del primer examen. 

\bigskip

\noindent\textbf{Palabras clave:} Clustering, CodeBERT, Embeddings, Enseñanza, Python, Programación.